\documentclass[10pt]{amsart}

\usepackage{aux/mainmeta}
\date{\today}

\begin{document}

\maketitle


These are notes for the Fall 2024 session of the \href{https://math.mit.edu/topology/babytop/index.html}{Babytop} seminar concerning \cite{HHR}.
All notes are copy-edited by Natalie Stewart, and they currently consist of writings of Isabel Longbottom. 
Individual writing attributions will be made on a per-lecture basis. 

\toc


% Set parskip after toc
\setlength{\parskip}{0.2em}



\newpage
\section{Isabel Longbottom: Intro to equivariant homotopy}
\begin{abstract}
    We give a brief introduction to equivariant homotopy theory. In the non-equivariant setting, homotopy theory is concerned with topological spaces up to weak equivalence. Before we can do equivariant homotopy theory, we need an equivariant notion of weak equivalence. Through a selection of examples, we present and try to motivate the relevant definitions. We then discuss Elmendorf's Theorem and how it gives us a very nice, concrete model for the $\infty$-category of $G$-spaces as presheaves on the orbit category. We conclude by saying a few words about equivariance in families.
\end{abstract}


\input{tex/9_24.tex}
 
\printbibliography
\end{document}
