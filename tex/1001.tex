\begin{abstract}
  We introduce $\ZZ$-graded equivariant cohomology theories on $G$-spaces with coefficients in a coefficient system. 
  We will look at an example computation using cellular cochains. 
  We define $G$-spectra to be the stabilization of $G$-spaces with respect to finite orthogonal $G$-representations. They represent $\RO(G)$-graded cohomology theories, where Mackey functors now play the role of the coefficients. 
\end{abstract}

The following notes were livetexed by Natalie Stewart for a talk given by Eunice Sukarto on September 24, 2024, then edited by Eunice afterwards.
They are chiefly based on Blumberg's notes \cite{burnside} and section 3 of HHR \cite{HHR}.

We will follow the following outline:
\begin{enumerate}
  \item Talk about equivariant cohomology theories on $G$-spaces ($\ZZ$-graded)
    \begin{enumerate}
      \item coefficient systems
      \item Eilenberg Mac Lane $G$-spaces
      \item cellular (co)chains
    \end{enumerate}
  \item Do an example from HHR
  \item Cover $G$-spectra
    \begin{enumerate}
      \item Construction
      \item $\RO(G)$-graded cohomology theories
      \item Mackey functors
      \item Eilenberg Mac Lane $G$-spectra.
    \end{enumerate}
\end{enumerate}

Throughout this talk, $\cS$ will refer to spaces, $\cS_G$ will refer to $G$-spaces with weak equivalences inverted, $\cS_*$, pointed spaces, and
\[
  \cO_G = \cbr{G/H \mid H \subset G} \subset G\Top
\]
the \emph{Orbit category}.

\subsection{Z-graded equivariant cohomology theories}%
\subsubsection{Eilenberg Mac Lane $G$-spaces}%

Recall that Elmendorf's theorem constructs an equivalence
\[
  \cS_G \xrightarrow\sim \Fun(\cO_G^{\op}, \cS)
\]
sending $X$ to the functor $G/H \mapsto X^H$.
\begin{definition}
  If $\cC$ is a category, then \emph{coefficient systems in $\cC$} are objects of the category
  \[
    \Coeff^G \deq \Fun(\cO_G^{\op},\cC).
  \]
\end{definition}

If $X$ is a $G$-space, the equivariant homotopy groups $\upi_nX$ of $X$ form coefficient systems by
\[
  \upi_n(X)((H) = \pi_n^H X.
\]
In fact, given $\uA \in \Coeff^G(\Ab)$, we may define the \emph{eilenberg Mac Lane $G$-space}
\[
  \cO_G^{\op} \xrightarrow{\uA} \Ab \xrightarrow{K(-,n)} \cS;
\]
this is identified by the property
\[
  \upi_n(X) = \begin{cases}
    \uA & m = n \\ 
    0 & \mathrm{otherwise}. 
  \end{cases}
\]
Now, the usual argument shows that $\Omega K(\uA,n) \simeq K(\uA,n-1)$, so we get an infinite-loop object in $\cS_G$.
These give examples of the following...
\begin{definition}
  A \emph{$\ZZ$-graded cohomology theory on $\cS_{G,*}$} is a functor $H\colon h \cS_{G,*}^{\op} \rightarrow \Ab^{\ZZ}$ with a natural isomorphism
  \[
    H^{\bullet + 1}(\Sigma -) \simeq H^\bullet(-)
  \]
  satisfying the Eilenberg Steenrod-axioms.
\end{definition}
Brown representability constructs representing $\infty$-loop $G$-spaces for $\ZZ$-graded cohomology theories.
\begin{example}
  We may define \emph{equivariant singular cohomology}
  \[
    H^*_G(X;\uA) \deq \pi_* \Map^G(X,H\uA),
  \]
  where $H\uA$ is the infinite loop $G$-space corresponding to $K(\uA,-)$.
\end{example}

\subsubsection{Cellular cochains}%
Let $X$ be a $G$-CW-complex.
We set the notatation $X_{n}$ for
\[
  X^{(n)}/X^{(n-1)} \simeq X_{n+} \wedge S^n,
\] 
where $X^(n)$ is the $n$th skeleton of $X$.
We define the cellular chain
\[
  C_n^{\mathrm{cell}}(X;\uA) = \pi_n^G(H\uA \wedge X^{(n)}/X^{(n-1)}) = \pi_0^G(H\uA \wedge X_{n+1})
\]
and cellular cochain
\[
  C^n_{\mathrm{cell}}(X;\uA) = [X^{(n)}/X^{(n-1)},H\uA] = [X_{n+} \wedge S^n, H\uA].
\]
These are (co)chain complexes with boundary maps defined in the usual way.
As in the non-equivariant case, it turns out that $H_*^G(X;\uA)$ is the cohomology of the cellular cochains.
\begin{lemma}\label{blah lem}\cite[Example 3.9]{HHR}
  Let $X$ be a $G$-CW-complex and $\uA$ a constant coefficient system.
  Then, there is a natural equivalence
    \[
      H^*_G(X;\uA) \simeq H^*(X/G,A)
    \]
\end{lemma}
\begin{proof}[Proof sketch]
  The proof follows by noting that the equivariant cellular cochains are invariants in the cellular cochains with the $G$-action.
\end{proof}

We will need the following as an input to the main computation.
\begin{lemma}\cite[Example 3.19]{HHR}\label{lem:3.19}
    Let $G$ be a nontrivial group, $\rho$ its regular representation, $n\geq 1$, and $\uA$ a constant coefficient system. Then 
    \[H_G^0(S^{n(\rho-1)};\uA) =  H_G^1(S^{n(\rho-1)};\uA)= 0.\]
\end{lemma}
\begin{proof}
    By Lemma \ref{blah lem}, 
    \[H_G^i(S^{n(\rho-1)};\uA) \cong H^i(S^{n(\rho-1)}/G;A).\]
    We have
    \[S^{n(\rho-1)} \simeq \Sigma S(n(\rho-1))\]
    as $G$-spaces, where $S(n(\rho-1))$ is the unit sphere, so the orbit space is also a suspension. If $\dim n(\rho-1)\geq 2$ i.e. $|G|>2$ or $G=C_2$ and $n\geq 2$, $S(n(\rho-1))$ is connected, hence so is its orbit space. If $G=C_2$ and $n=1$, $\rho$ is the sign representation, so $S(\rho-1)\simeq C_2$ and $S(\rho-1)/G\simeq *$ is connected. Thus, in all cases, $S(n(\rho-1))/G$ is connected and $S^{n(\rho-1)}/G$ is simply connected, giving us the result.
\end{proof}

We are now ready to do the main computation.
\begin{proposition}\cite[Proposition 3.20]{HHR}
  Let $G$ be a nontrivial group and $n \geq 0$.
  Then, unless $G = C_3$, $i = 3$ and $n = 1$, we have
  \[
    H_G^i(S^{n\rho};\uZZ) = 0; \hspace{50pt} 1 \leq i \leq 3.
  \]
  For the exceptional case, we have
  \[
    H_{C_3}^3(S^\rho;\uZZ) = \ZZ.
  \]
\end{proposition}
\begin{proof}
  The case $n=0$ is clear since $S^0$ only has cells in dimension 0, so we may assume that $n>0$.

  In general, if $n = 0$, this is easy, so we choose $n > 0$.
  We have
  \[H_G^i(S^{n\rho};\uZZ) \simeq H_G^{i-n}(S^{n(\rho-1)};\uZZ).\] 
  This, together with Lemma \ref{lem:3.19} and the connectivity of $S^{n(\rho-1)}$ implies that $H_G^i(S^{n\rho};\uZZ) =0$ for $i\leq n+1$. If either $n + 1 \geq 3$ or $n=1$ and $i=1,2$, we are done by dimension counting.
  
  So, we are left with the case $n = 1$ and $i = 3$.
  Write $X = S^{\rho - 1}/G$ and note that $X$ is simply connected by Lemma \ref{lem:3.19}, so UCT constructs an injection
  \[
    H^2(X;\ZZ) \hookrightarrow H^2(X;\QQ).
  \]
  Since $\QQ$ has characteristic 0, we have
  \[
    H^*(Y/G;\QQ) \simeq H^*(Y;\QQ)^G. 
  \] for any space $Y$ with $G$-action.
  If $\abs{G} \neq 3$, we get $H^2(X;\ZZ) = 0$.
  If $G=C_3$, then $H^2(S^{\rho-1};\ZZ)^G = \QQ$, so $H^2(X;\ZZ) = \ZZ$.
\end{proof}

We now move on to $G$-spectra.

\subsection{\texorpdfstring{$G$}{G}-spectra}%
\def\SW{\mathrm{SW}}
From here on, we say ``$G$-representation'' to mean finite dimensional orthogonal real representations of $G$.
Given a pointed $G$-spaces $X \in \cS_{G,*}$, we may write
\[
  \Sigma^V X \deq S^V X; \hspace{40pt} \Omega^V X = \Map(S^V,X); \hspace{40pt} \pi_V^G X = \pi_0^G \Omega^V X.
\]
We define the \emph{$G$-Spanier whitehead category} $\SW^G$ to have objects finite pointed $G$-CW complexes and morphisms
\[
  \cbr{X,Y}^G = \colim_V \Map^G(\Sigma^V X, \Sigma^V Y).
\]
This is not stable...
\begin{definition}
  The \emph{$\infty$-category of $G$-spaces} is the limit
  \[
    \Sp_G \deq \lim \prn{\cdots \xrightarrow{\Omega^{\rho}} \cS_{G,*} \xrightarrow{\Omega^{\rho}} \cS_{G,*}}.
  \]
\end{definition}
Since every $G$-representation embeds into some $n\rho$, this inverts $S^V$ for all $G$-reps $V$.
Given $X \in \Sp_G$, we let
\[
  \pi_k^G X = \colim_{V \gg 0} \pi_{V +k}^H X_V.
\]
We will write $\RO(G)$ for the Grothendieck group of the monoid of $G$-representations.
An \emph{$\RO(G)$-graded cohomology theory} will refer to a functor
\[
  E\colon  \RO(G) \times h\cS_{G,*}^{\op} \rightarrow \Ab
\]
with a natural equivalence
\[
  E^{V}(X) \simeq E^{V \oplus W}(\Sigma^W X)
\]
for each $V$,$W$, and satisfying the Eilenberg-Steenrod axioms.
In general, given a $G$-spectrum $E$, we get one of these via
\[
  E^V(X) \deq [\Sigma^{\infty} X, S^V\wedge E];
\]
$\RO(G)$-graded Brown representability says that this gives \emph{every} $\RO(G)$-graded chomology theory.

Since $\Sigma^V$ and $\Omega^V$ are inverses to each other in $h\Sp_G$, a $\ZZ$-graded cohomology theory on $\Sp_G$ always extends to an $\RO(G)$-graded cohomology theory. This is not true for $\cS_{G,*}$.
In general, $H^*(X;\uA)$ extends to an $\RO(G)$-graded cohomology theory on $\cS_{G,*}$ if and only if $\uA$ is the underlying coefficient system of a Mackey functor in the following sense.

\subsection{Mackey functors}%
\begin{definition}
  Let $\cC$ be a small category with finite limits and finite coproducts.
  Then, we can form the category $\Span(\cC)$ with objects given by $\Ob(\cC)$ and morphisms $X \rightarrow Y$ the isomorphism-classes of spans
  \[
    X \leftarrow R \rightarrow Y.
  \]
  We define $\Span^+(\cC)$ to be the completion of $\Span(\cC)$ to an additive category;
  that is, its hom abelian groups are the grothendieck groups of the hom commutative monoids on $\Span(\cC)$.
  We define the \emph{additive Brunside category}
  \[
    B_G \deq \Span^+(\FF_G),
  \]
  where $\FF_G$ denotes finite $G$-sets.
\end{definition}
\begin{definition}
  A \emph{Mackey functor} is a product-preserving functor
  \[
    B_G^{\op} \rightarrow \cA,
  \]
  where $\cA$ is an additive category.
  We set the notation
  \[
    \Mack^G(\cA) \deq \Fun^{\times}(B_G^{\op},\cA);
  \]
  this corresponds to a pair of covariant and contravariant functors $\cO_G \rightarrow \cA$ and $\cO_G^{\op} \rightarrow \cA$ with the same objects, and satisfying a \emph{Beck-Chevalley condition}.
\end{definition}
\begin{example}
  Some examples include the following:
  \begin{itemize}
    \item $\upi_* X$, where $X \in \Sp_G$.
    \item the ``constant'' Mackey functor $\uA$ 
    \item the ``Burnside'' Mackey functor $\uA(G)$, where $\uA(G)(H)$ is the group ring of finite $H$-sets with restriction and induction as contravariant and covariant functoriality.
  \end{itemize}
\end{example}

\begin{claim}
  Every Mackey functor is $\pi_0 H\uM$ for some $G$-spectrum $H\uM$ characterized by the property
  \[
    \upi_n H\uM = \begin{cases}
      \uM & n=0 \\ 
      0 & \mathrm{otherwise} 
    \end{cases}
  \]
\end{claim}
\begin{proof}
  We may construct this by constructing a $\ZZ$-graded cohomology theory
  \[
    C_n(X) \deq \pi_n(X^{(n)}/X^{(n-1)}); \hspace{40pt} C^n(X;\uM) \deq [C_n(X), \uM]
  \]
  with hom taken in Mackey functors.
  This is a chain complex of Abelian groups;
  Brown representability then lifts this to a $G$-spectrum, and an explicit computation of cohomology of $\SS$ verifies that $H\uM$ has the correct homotopy groups.
\end{proof}

